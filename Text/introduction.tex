\section{Introduction}

The Internet of things (IoT) is increasingly becoming part of our
everyday life. In order to protect the processed data and the privacy of the users, it is important to implement dedicated security measures in IoT devices and networks. \par
Information-centric networking (ICN) is an approach to evolve the Internet infrastructure by introducing uniquely named data as a core Internet principle. Data becomes independent from location, application, storage, and means of transportation, enabling in-network caching and replication. The expected benefits are improved efficiency, better scalability with respect to information/bandwidth demand and better robustness in challenging communication scenarios.\par
ICN offers many features including caching, computing and storage, mobility, context-aware networking and support for ad hoc networking features, all of which have to be realized in an application-specific means in the context of IP-IoT. These compelling features enable a distributed and intelligent data distribution platform to support heterogeneous IoT services with features like device bootstrapping with minimal configuration, simpler protocols to aid self-organizing among the IoT elements, natural support for compute and caching logic at strategic points in the network.\par
Since data in information-centric networking is delocalized and
need not be retrieved via an end-to-end transport stream but
instead with hop-wise replication and in-network caching, it facilitates information dissemination in the IoT environment, and relaxs the demand for continued connectivity.\par
ICN concepts can be applied to different layers of the protocol stack: name-based data access can be implemented on top of the existing IP infrastructure, e.g., by providing resource naming, ubiquitous caching and corresponding transport services, or it can be seen as a packet-level internetworking technology that would cause fundamental changes to Internet routing and forwarding.\par
With ICN, new security models are needed. Today’s host-centric trust model - retrieving data from a trusted server via a secure connection - no longer applies. Instead, security/trust/identity functions are bound to the information objects themselves, employing signed objects and and ensuring name-data integrity. Moreover, not all objects will be universally accessible, requiring authorization and scoping mechanisms.\par
The heterogeneity of both network equipment deployed and services offered by IoT networks leads to a large variety of data, services and devices. While using a traditional host-centric architecture, only devices or their network interfaces are named at the network level, leaving to the application layer the task to name data and services. This causes different applications to use different naming schemes, and no consistent mapping from application layer names to network names exist. In many common applications of IoT networks, data and services are the main goal, and ICN provides an intuitive way to name those in a way that can be utilized on the network layer as well. Communication with a specific device is often secondary, but when needed, the same ICN naming mechanisms can be used. The network distributes content and provides a service, instead of only sending data between two named devices. In this context, data content and services can be provided by several devices, or group of devices, hence naming data and services is often more important than naming the devices. This naming mechanism also enables self-configuration of the IoT system.\par
ICN provides data integrity through Name-Data Integrity, i.e., the guarantee that the given data corresponds to the name with which it was addressed. Signature-based schemes can additionally provide data authenticity, meaning establishing the origin, or provenance, of the data, for example, by cryptographically linking a data object to the identity of a publisher. Confidentiality can be handled on a per object basis based on keys established at the application level. All of this means that the actual transmission of data does not have to be secured as the same security mechanisms protect the data after generation until consumed by a client, regardless of whether it is in transit over a communication channel or stored in an intermediate cache. In an ICN network, each individual object within a stream of immutable objects could potentially be retrieved from a cache in a different location. Having a trust relationship with each of these different caches is not realistic. Through Name-Data Integrity, ICN automatically guarantees data integrity to the requester regardless of the location from where it is delivered. The Object Security model also ensures that the content is readily available in a secure state in the device constraints are severe enough that it is not able to perform the required cryptographic operations for Object Security, it may be possible to offload this operation to a trusted gateway to which only a single secure channel needs to be established. 